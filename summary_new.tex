We  obtained {\em Spitzer}/IRS spectral maps of 12 regions within M31 covering wavelengths 5--21~$\mu$m. 
The spectra from those regions, except for the nucleus, are similar to spectra obtained from other nearby  star-forming galaxies. 
Early  ISOCAM observations towards 4 regions of M31 showing a suppression 
of the 6--8~$\mu$m features' strength and an enhancement of  the 11.3~$\mu$m feature intensity and FWHM \citep{1998Cesarsky} were likely affected by the background subtraction methods applied.

We recover the well-known strong correlation between the 6.2/11.2 and 7.7/11.2 $\mu$m PAH intensity ratios that is governed by the PAH charge balance.  This is another indicator that the PAH emission in all regions but the nucleus is typical. The PAH EQWs in M31 regions do not show a clear decreasing trend with increasing radiation hardness, but are consistent with previous 
results from other nearby galaxies. The distribution of PAH EQWs with metallicity is well within the range of the starburst galaxy sample of \citet{Engelbracht_2008}. 
We did not have enough data from low-metallicity regions of M31 to observe the decreasing trend of EQWs at low metallicities which is visible in other galaxies.

Different spectral features (11.2~$\mu$m PAH emission, [NeIII] 15.5~$\mu$m line emission, silicate emission) show distinct spatial distributions in the nuclear region. The mid-infrared spectrum from the nucleus of M31 shows a strong blue continuum and clear silicate emission. This nuclear spectrum is largely dominated by the stellar component with little contribution from the nucleus itself. In addition to a blue continuum and silicate emission, the mid-infrared spectrum of a region north of the nucleus  (15\arcsec ~off-nucleus) exhibits suppressed 6--8~$\mu$m PAH features and normal 11.3 and 17.0 $\mu$m PAH emission.
This atypical PAH emission is only seen elsewhere towards some elliptical galaxies and low luminosity AGNs \citep[e.g.][]Smith:2007lr, Kaneda:08, Vega:10}. 
{\bf  my suggestion for these last few sentences - PB}
%The nuclear spectrum is similar to that of six other nearby galaxies known to have low luminosity AGN activity. This is discussed in the context of 
%the suggestion by \citet{Smith:2007lr} that low $L(7.7\mu{\rm m})/L(11.3\mu{\rm m})$ is an indicator of low luminosity AGN. 

%The 12~$\mu$m luminosity can be used to estimate a bolometric luminosity for the M31 nucleus of $1.6\times 10^{40}$~erg~s$^{-1}$, well within the `low-luminosity' classification, but well above the value estimated from the X--ray flux.
