Mid-infrared spectra provide a unique diagnostic tool to understand the physical conditions in the interstellar medium of galaxies. 
The rich range of spectral features (Polycyclic Aromatic Hydrocarbons (PAHs), atomic fine structure lines (e.g. Ne, S) and the
amorphous silicate feature centred at 9.7~$\mu$m) provide information on dust properties, radiation field and star formation. 
With the advent of infrared space telescopes, such as the Infrared Space Observatory (ISO, \citealt{Kessler1996}) and 
the {\em Spitzer} Space Telescope \citep{spitzer2004}, we have been able to well explore the infrared emission from galaxies. 

PAHs are known as the main carrier of the ubiquitous mid-IR emission bands (e.g. \citealt{Allamandola1989}, 
\citealt{Tielens2008}). They are large hydrocarbon molecules consisting of $\sim$50--100 carbon atoms. 
The main PAH features are seen at 3.3, 6.2, 7.7, 8.6, 11.3 and 12.7~$\mu $m (e.g.\citealt{Mattila1996}, \citealt{Peeters2002}), 
and these bands are due to the vibrational de-excitation of PAH molecules  through bending and stretching modes of C-H and C-C bonds \citep{Tielens:2005lr}. 
The 6 to 8 micron features are thought to originate mostly from ionized PAHs and the 3.3, 11.3, 12.7 and 17.1~$\mu$m 
emission bands from neutral PAHs \citep{Peeters2002}. 


The relative strengths of the PAH features do not vary much within normal-luminosity galaxies \citep{Smith:2007lr} or within 
massive starburst galaxies \citep{Brandl2006}. But they do change significantly close to active galactic nuclei where the 
strength of PAHs gets weaker (\citealt{Roche1991}, \citealt{Smith:2007lr}). \citet{Smith:2007lr}  found that the mid-IR 
spectra from weak AGNs show suppressed 6 to 8~$\mu$m PAH features but are bright at 11.3~$\mu$m. 
A possible explanation for this behaviour is that AGNs alter the grain composition by selective destruction of small ionized PAHs. 
ISOCAM spectro-imaging observations of M31\citep{1998Cesarsky} showed that four regions including the nucleus and bulge 
of this galaxy have very odd PAH spectra, bright at 11.3 and 12.7~$\mu$m but lacking the usual 6.2, 7.7, and 8.6 micron bands. 
Investigating this unusual PAH emission was the main motivation for the work described in this paper. 


Previous studies of nearby galaxies indicate that metallicity and radiation hardness correlate with PAH equivalent widths (EQWs). 
\citet{Smith:2007lr} and \citet{Engelbracht_2008} showed that PAH EQWs in nearby star forming galaxies  decrease with increasing radiation hardness. 
But  \citet{Brandl2006} found no correlation within their starburst sample.  With metallicity, PAH EQWs show an anti-correlation 
in star-forming galaxies \citep{Marble_2010}. This variation of PAHs among galaxies has also been observed within H~{\sc ii} regions 
of a single galaxy (M101) by \citet{Gordon:2008lr}. But there are no other investigations done on a single star-forming galaxy with 
sufficiently high resolution to see whether the correlations mentioned above hold within a galaxy similar to the Milky Way.


The amorphous silicate feature at 9.7 $\mu$m is another aspect of the mid-IR spectra of galaxies. Depending on the presence of silicate 
absorption or emission, the overall shape of the mid-IR spectra, that is the continuum and the PAH intensities, can change. \citet{Spoon2007} 
classified infrared galaxies based on the equivalent width of the 6.2 $\mu$m PAH feature and the strength of the 9.7 $\mu$m silicate feature. 
They  found galaxies spread along two distinct branches: one of AGN and starburst-dominated spectra and one of deeply obscured 
nuclei and starburst-dominated spectra. The first branch is horizontal along emission or weak-absorption of the silicate feature and show no 
correlation with the 6.2~$\mu$m PAH feature (Figure~1 in \citet{Spoon2007}). Silicate emission at 9.7~$\mu$m has also been observed in both 
Seyfert~1 and Seyfert~2 galaxies \citep{Mason2009}. Therefore it is important to study the infrared spectra from nuclei with higher resolution 
to understand this silicate feature and how it reflects the physical structure of the nucleus. 

M31 with its proximity ($\sim$780 kpc) and rich observational databases provides the most detailed view of a star forming galaxy similar 
to the Milky Way. The active star forming ring \citep{Barmby2006lr} provides evidence of abundant PAHs in M31. 
We employed mid-IR spectral maps from the {\em Spitzer}/Infrared Spectrograph (IRS) from 12 regions of M31 for a further investigation of 
its infrared properties. This sample includes the nucleus, bulge, an active region in the star-forming ring (all previously observed by ISOCAM), and 9 
other regions chosen to cover a range of properties as described in Section~\ref{sect:irs_obs}. 
We obtained the processed version of ISOCAM observations of M31 and compare them with the IRS results in Section~\ref{sect:iso_vs_irs}. 
Section~\ref{sect:pah_ratios} discusses PAH intensity ratios.
In Section~\ref{sect:eqw_rh}, we investigate the relationship between PAH equivalent widths and radiation 
hardness and compare to that found by \citet{Engelbracht_2008} and \citet{Gordon:2008lr}. Metallicity and PAH EQWs are compared in 
Section~\ref{sect:eqw_met}, and Section~\ref{sect:nucleus} discusses the dust properties of the nucleus. 	
