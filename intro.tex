Mid-infrared spectra provide a unique diagnostic tool to understand the physical conditions in the interstellar medium of galaxies. 
The rich range of spectral features (Polycyclic Aromatic Hydrocarbons (PAHs), atomic fine structure lines (e.g. Ne, S) and the
amorphous silicate feature centred at 9.7~$\mu$m) provide information on dust properties, radiation field and star formation. 
With the advent of infrared space telescopes, such as the Infrared Space Observatory (ISO, \citealt{Kessler1996}) and 
the {\em Spitzer} Space Telescope \citep{spitzer2004}, we have begun to explore the detailed infrared emission from galaxies. 

PAHs are known as the main carrier of the ubiquitous mid-infrared emission bands (e.g. \citealt{Allamandola1989}, \citealt{puget89}).
They are large hydrocarbon molecules consisting of $\sim$50--100 carbon atoms \citep{Tielens2008}. 
The main PAH features are seen at 3.3, 6.2, 7.7, 8.6, 11.3 and 12.7~$\mu $m (e.g., \citealt{Mattila1996}, \citealt{Peeters2002}), 
and these bands are due to the vibrational de-excitation of PAH molecules  through bending and stretching modes of C-H and C-C bonds \citep{Tielens:2005lr}. 
The 6 to 8 micron features are thought to originate mostly from ionized PAHs and the 3.3 and 11.3~$\mu$m 
emission bands from neutral PAHs \citep{Peeters2002}. 


The relative strengths of the different PAH features do not show much spatial variation within normal-luminosity galaxies \citep{Smith:2007lr} or 
massive starburst galaxies \citep{Brandl2006}. However, feature ratios do change significantly close to active galactic nuclei where the overall
strength of PAHs also gets weaker (\citealt{Roche1991}, \citealt{Smith:2007lr}). \citet{Smith:2007lr}  found that the mid-infrared 
spectra from weak AGNs show suppressed 6 to 8~$\mu$m PAH features but are bright at 11.3~$\mu$m. 
One possible explanation for this behaviour is that AGNs alter the grain composition by selective destruction of small ionized PAHs. 

 
Previous studies of nearby galaxies indicate that metallicity and radiation hardness both affect PAH equivalent widths (EQWs). 
\citet{Smith:2007lr} and \citet{Engelbracht_2008} showed that PAH EQWs in nearby star-forming galaxies  decrease with increasing radiation hardness,
although  \citet{Brandl2006} found no such correlation within their starburst sample.  With metallicity, PAH EQWs show an anti-correlation 
in star-forming galaxies \citep{Marble_2010}. This variation of PAHs among galaxies has also been observed within H~{\sc ii} regions 
of a single galaxy (M101) by \citet{Gordon:2008lr}. But there are no other investigations done on a single star-forming galaxy with 
sufficiently high resolution to see whether the correlations mentioned above hold within a galaxy similar to the Milky Way.

The amorphous silicate feature at 9.7~$\mu$m is another aspect of the mid-infrared spectra of galaxies and in particular their nuclei.  \citet{Spoon2007} 
classified infrared galaxies based on the equivalent width of the 6.2~$\mu$m PAH feature and the strength of the 9.7~$\mu$m silicate feature. 
They  found galaxies spread along two distinct branches: one in which silicate absorption strength was anti-correlated with PAH
equivalent width, and another in which the weak silicate feature strength did not depend on the 6.2~$\mu$m equivalent width.
Silicate {\em emission} at 9.7~$\mu$m has also been observed in both Seyfert~1 and Seyfert~2 galaxies 
and can be used to constrain the geometry and structure of the emitting nuclear region \citep{Mason2009}.



M31 with its proximity \citep[$785\pm25$ kpc; ][]{Mcc2005} and rich observational databases provides the most detailed view of a star forming galaxy similar 
to the Milky Way. The active star forming ring visible in 8~$\mu$m  {\em Spitzer}/IRAC images \citep{Barmby2006lr} provides evidence of abundant PAHs in M31. 
However, ISOCAM spectro-imaging observations of M31\citep{1998Cesarsky} showed that four regions including the nucleus and bulge 
of this galaxy have very odd PAH spectra, bright at 11.3~$\mu$m but lacking the usual 6.2, 7.7, and 8.6 micron bands. 
Investigating this unusual PAH emission was the main motivation for the work described in this paper. 
The centre of M31 has a complicated physical structure. It hosts a very inactive supermassive black hole with a mass of 
$0.7-1.4 \times 10^8$~M$_{\sun}$ \citep{Bacon2001, Bender2005} and also has a lopsided nuclear disk  with two stellar 
components \citep{Lauer1993} and an A-star cluster \citep{Bender2005}. While M31's nucleus is known to be inactive from an 
X--ray perspective \citep{Li2011}, mid-infrared indicators of its nuclear activity, such
as infrared excess or spectral features of silicates,  have received relatively little attention. 
The higher spatial resolution available in observations of  a very nearby galaxy like M31, compared to 
luminous, distant objects such as ultra-luminous infrared galaxies \citep{Spoon2007} or nearby Seyferts \citep{Mason2009},
makes exploring its mid-infrared spectrum worthwhile.


We employed mid-infrared spectral maps from the {\em Spitzer}/Infrared Spectrograph (IRS) from 12 regions of M31 for a further investigation of 
its infrared properties. This sample includes the nucleus, bulge, an active region in the star-forming ring (all previously observed by ISOCAM), and 9 
other regions chosen to cover a range of properties as described in Section~\ref{sect:irs_obs}. 
We obtained the processed version of ISOCAM observations of M31 and compare them with the IRS results in Section~\ref{sect:iso_vs_irs}. 
Section~\ref{sect:pah_ratios} discusses PAH intensity ratios.
In Section~\ref{sect:eqw_rh}, we investigate the relationship between PAH equivalent widths and radiation 
hardness and compare to that found by \citet{Engelbracht_2008} and \citet{Gordon:2008lr}. Metallicity and PAH EQWs are compared in 
Section~\ref{sect:eqw_met}, and Section~\ref{sect:nucleus} discusses the dust properties of the nucleus. 	
