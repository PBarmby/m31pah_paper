We  obtained {\em Spitzer}/IRS spectral maps of 12 regions within M31 covering wavelengths 5--21~$\mu$m. 
The spectra from those regions, except for the nucleus, are similar to spectra obtained from other nearby  star-forming galaxies. 
However, our spectra are inconsistent with previous ISOCAM observations of M31 \citep{1998Cesarsky} which reported a suppression 
of the 6--8~$\mu$m features and an enhancement of 11.3~$\mu$m feature towards 4 regions. 
Our IRS spectra for three of these regions do not show this unusual behaviour and neither do spectra extracted from the reprocessed version
of the ISOCAM data.
We conclude that the earlier results based on ISOCAM data were likely affected by the background subtraction methods applied.

The equivalent widths of PAH features in M31 regions showed a decreasing trend with increasing radiation hardness, consistent with previous 
results from other nearby galaxies. The distribution of PAH EQWs with metallicity was well within the values from the starburst galaxy sample of \citet{Engelbracht_2008}. 
We did not have enough data from low-metallicity regions of M31 to observe the decreasing trend of EQWs at low metallicity values which is visible in other galaxies.

Mid-infrared spectra from near the nucleus of M31 show either suppressed 6--8~$\mu$m features and a strong 11.3~$\mu$m feature
(off-nucleus, GIVE DISTANCE) or silicate emission around 9.7~$\mu$m  (on-nucleus). The off-nucleus region spectrum is similar to that of
six other nearby galaxies known to have low-luminosity AGN activity. This could strengthen the
suggestion by \citet{Smith:2007lr} that low $L(7.7\mu{\rm m})/L(11.3\mu{\rm m})$ is an indicator of low luminosity AGN,
but this feature ratio could also be due to a lack of ionized PAHs. The nuclear silicate emission is another possible AGN indicator
and should be further explored. % UGH! Do something with this last setence.  
 