We have obtained {\em Spitzer}/IRS spectral maps of 12 regions of M31 covering wavelengths 5--21~$\mu$m. 
The spectra from those regions, except the nucleus, agree with spectra obtained from other nearby  star forming galaxies. 
However, they are inconsistent with previous ISOCAM observations of M31 \citep{1998Cesarsky} reporting a suppression 
of the 6--8~$\mu$m features and an enhancement of 11.3~$\mu$m feature towards 4 regions. 
Our IRS spectra for three of these ISOCAM regions do not show this unusual behaviour. 
ISOCAM data corresponding to those three regions in common between our study and that of \citep{1998Cesarsky} 
were obtained from the ISO archive. These data were reprocessed in 2005 to remove contamination by stray light and zodiacal emission; 
spectra extracted from these newly processed data could not reproduce previous results. 
Therefore we conclude that the earlier results based on ISOCAM data were affected by the background subtraction methods applied to overcome the contamination.

The equivalent widths of PAH features in M31 regions showed a decreasing trend with increasing radiation hardness, consistent with previous 
results from other nearby galaxies. The PAH EQWs versus metallicity data points were well within the values from the starburst galaxy sample of \citet{Engelbracht_2008}. 
We did not have enough data from low-metallicity regions of M31 to observe the decreasing trend of EQWs at low metallicity values which is visible in other galaxies.

The mid-infrared spectrum from the nucleus was compared with that of six other galaxies which are known to have AGN activity. 
All have a similar spectral shape and present suppressed 6--8~$\mu$m features and a strong 11.3~$\mu$m feature. 
\citealt{Smith:2007lr} says that this type of spectrum can be used to identify a low luminosity AGN. 
Therefore we argue that the centre of M31 hosts a low-luminosity AGN, supporting that argument. %TODO: sentence needs work
The spectrum obtained from the centre region of the nucleus had a strong silicate emission around 9.7~$\mu$m which is also evidence for the presence of a low-luminosity AGN.
