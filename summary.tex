We  obtained {\em Spitzer}/IRS spectral maps of 12 regions within M31 covering wavelengths 5--21~$\mu$m. 
The spectra from those regions, except for the nucleus, are similar to spectra obtained from other nearby  star-forming galaxies. 
Early  ISOCAM observations  towards 4 regions of M31 showing a suppression 
of the 6--8~$\mu$m features and an enhancement of  the 11.3~$\mu$m feature  \citep{1998Cesarsky} 
were likely affected by the background subtraction methods applied.

The PAH intensities in M31 regions show a decreasing trend with increasing radiation hardness, consistent with previous 
results from other nearby galaxies. The distribution of PAH EQWs with metallicity is well within the range of the starburst galaxy sample of \citet{Engelbracht_2008}. 
We did not have enough data from low-metallicity regions of M31 to observe the decreasing trend of EQWs at low metallicities which is visible in other galaxies.

Mid-infrared spectra from near the nucleus of M31 show either suppressed 6--8~$\mu$m features and a strong 11.3~$\mu$m feature
(15\arcsec off-nucleus) or silicate emission around 9.7~$\mu$m  (on-nucleus). 
The nuclear spectrum is similar to that of six other nearby galaxies known to have low-luminosity AGN activity. This could strengthen the
suggestion by \citet{Smith:2007lr} that low $L(7.7\mu{\rm m})/L(11.3\mu{\rm m})$ is an indicator of low luminosity AGN,
but this feature ratio could also be due to a lack of ionized PAHs. The nuclear silicate emission is another possible AGN indicator.
The 12~$\mu$m luminosity can be used to estimate a bolometric luminosity for the M31 nucleus of $1.6\times 10^{40}$~erg~s$^{-1}$,
well within the `low-luminosity' classification, but well above the value estimated from the X--ray flux.
