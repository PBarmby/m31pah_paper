\begin{abstract}
We present {\sl Spitzer}/Infrared Spectrograph 5 to 21 $\mu$m spectroscopic maps towards 12 regions in the Andromeda galaxy(M31). These regions include the nucleus, bulge, active region in the star-forming ring, and 9 other regions chosen to cover a range of mid-to-far-infrared colours. Our IRS observations did not recover the unusual PAH ratios (suppressed 6 - 8 $\mu$m features and an enhanced 11.3 $\mu$m feature) seen in ISOCAM spectro-imaging observations. PAH feature ratios (6.2 $\mu$m and the 7.7 $\mu$m features compared to the 11.3 $\mu$m feature) measured from our extracted M31 spectra are consistent with these seen in other nearby galaxies. The equivalent widths of the main PAH features decrease with increasing radiation hardness, consistent with that observed for other nearby spiral and starburst galaxies. The nucleus does not show any PAH emission except the 11.3 $\mu$m feature, but does show strong silicate emission at 9.7 $\mu$m. Both of these characteristics provide evidence for a low luminosity AGN in M31.
\end{abstract}

\begin{keywords}
techniques: spectroscopic - ISM: PAH molecules - galaxies: starburst - infrared: Dust emission.
\end{keywords}

