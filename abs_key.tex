\begin{abstract}
We present {\sl Spitzer}/Infrared Spectrograph 5--21~$\mu$m spectroscopic maps towards 12 regions in the Andromeda galaxy (M31). 
These regions include the nucleus, bulge, an active region in the star-forming ring, and 9 other regions chosen to cover a range of mid-to-far-infrared colours. 
Our observations did not reproduce the suppressed 6--8~$\mu$m features and enhanced 11.3~$\mu$m feature intensity and FWHM seen with the ISOCAM instrument on the Infrared Space Observatory.
In line with previous results, PAH feature ratios (6.2~$\mu$m and 7.7~$\mu$m features compared to the 11.2~$\mu$m feature) measured from our extracted M31 spectra, except the nucleus, strongly correlate. The equivalent widths of the main PAH 
features decrease with increasing radiation hardness, consistent with that observed for other nearby spiral and starburst galaxies. 
The nucleus does not show any PAH emission but does show strong silicate emission at 9.7~$\mu$m. Furthermore, different spectral features (11.3~$\mu$m PAH emission, silicate emission and [NeIII] 15.5~$\mu$m line emission) have distinct spatial distributions in the nuclear region: the silicate emission is confined towards the nuclear centre (19--27pc) while the PAH emission peaks 15\arcsec north of the centre. At this position, the PAH emission is atypical with strong emission at 11.2~$\mu$m and at 15-20~$\mu$m but suppressed emission at 6--8~$\mu$m. We also find that the silicate emission profile is redshifted and broader compared to that observed towards the Galactic centre. 
Both the silicate emission and atypical PAH emission provide evidence for a low luminosity active galactic nucleus in M31.
\end{abstract}

\begin{keywords}
galaxies: individual: M31 --
galaxies: ISM --
galaxies: nuclei --
infrared: ISM --
ISM:  molecules -- 
ISM: lines and bands
\end{keywords}

