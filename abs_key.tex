% MLNA comments
%Abstract: 
%
%We present observations of the mid-IR spectra obtained from Spitzer-IRS, covering the wavelength range from 5 to 21 microns towards 12 regions in M31 
%
%-->  We present {\sl Spitzer}/Infrared Spectrograph 5--21\,$\mu$m spectra towards 12 regions in M31
%
%acronym ISOCAM not defined at first use
%
%ISOCAM spectro-imaging observations of M31 showed unusual PAH feature ratios; however, the spectra that we obtained show PAH emission between 6 - 8 microns disagreeing with the previous results from ISOCAM observations.
%
%--> Our spectra reveal PAH emission between 6 and 8 microns that is inconsistent with earlier ISO/ISOCAM observations by [REF HERE] that showed unusual PAH feature ratios.
%
%important: Later on you refer to a re-reduction of the ISOCAM data.  If that's already out there, and it refutes Cesarsky et al, then there's really no need to refer to that paper as a result that we need to confront...somebody else already did it.  In which case, the motivational part of the paper ought not to present Cesarsky et al as something that's still current.


\begin{abstract}
We present {\sl Spitzer}/Infrared Spectrograph 5--21~$\mu$m spectroscopic maps towards 12 regions in the Andromeda galaxy (M31). 
These regions include the nucleus, bulge, an active region in the star-forming ring, and 9 other regions chosen to cover a range of mid-to-far-infrared colours. 
PAH feature ratios (6.2~$\mu$m and 7.7~$\mu$m features compared to the 11.3~$\mu$m feature) 
measured from our extracted M31 spectra are consistent with these seen in other nearby galaxies. 
Our  observations did not recover the unusual PAH ratios (suppressed 6--8~$\mu$m features and an enhanced 11.3~$\mu$m feature) seen in 
spectro-imaging observations with the ISOCAM instrument on the Infrared Space Observatory. 
The equivalent widths of the main PAH 
features decrease with increasing radiation hardness, consistent with that observed for other nearby spiral and starburst galaxies. 
The nucleus does not show any PAH emission except for the 11.3~$\mu$m feature, but does show strong silicate emission at 9.7~$\mu$m. 
Both of these characteristics provide evidence for a low luminosity active galactic nucleus in M31.
\end{abstract}

\begin{keywords}
galaxies: individual: M31 --
galaxies: ISM --
galaxies: nuclei --
infrared: ISM --
ISM:  molecules -- 
ISM: lines and bands
\end{keywords}

